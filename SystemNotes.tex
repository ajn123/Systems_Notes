%%%%%%%%%%%%%%%%%%%%%%%%%%%%%%%%%%%%%%%%%
% Daily Laboratory Book
% LaTeX Template
%
% This template has been downloaded from:
% http://www.latextemplates.com
%
% Original author:
% Frank Kuster (http://www.ctan.org/tex-archive/macros/latex/contrib/labbook/)
%
% Important note:
% This template requires the labbook.cls file to be in the same directory as the
% .tex file. The labbook.cls file provides the necessary structure to create the
% lab book.
%
% The \lipsum[#] commands throughout this template generate dummy text
% to fill the template out. These commands should all be removed when 
% writing lab book content.
%
% HOW TO USE THIS TEMPLATE 
% Each day in the lab consists of three main things:
%
% 1. LABDAY: The first thing to put is the \labday{} command with a date in 
% curly brackets, this will make a new page and put the date in big letters 
% at the top.
%
% 2. EXPERIMENT: Next you need to specify what experiment(s) you are 
% working on with an \experiment{} command with the experiment shorthand 
% in the curly brackets. The experiment shorthand is defined in the 
% 'DEFINITION OF EXPERIMENTS' section below, this means you can 
% say \experiment{pcr} and the actual text written to the PDF will be what 
% you set the 'pcr' experiment to be. If the experiment is a one off, you can 
% just write it in the bracket without creating a shorthand. Note: if you don't 
% want to have an experiment, just leave this out and it won't be printed.
%
% 3. CONTENT: Following the experiment is the content, i.e. what progress 
% you made on the experiment that day.
%
%%%%%%%%%%%%%%%%%%%%%%%%%%%%%%%%%%%%%%%%%

%----------------------------------------------------------------------------------------
%	PACKAGES AND OTHER DOCUMENT CONFIGURATIONS
%----------------------------------------------------------------------------------------

\documentclass[15pt,idxtotoc,hyperref,openany]{labbook} % 'openany' here removes the gap page between days, erase it to restore this gap; 'oneside' can also be added to remove the shift that odd pages have to the right for easier reading

\usepackage{listings}
\usepackage{color}

\definecolor{dkgreen}{rgb}{0,0.6,0}
\definecolor{gray}{rgb}{0.5,0.5,0.5}
\definecolor{mauve}{rgb}{0.58,0,0.82}

\lstset{frame=tb,
  language=C,
  aboveskip=3mm,
  belowskip=3mm,
  showstringspaces=false,
  columns=flexible,
  basicstyle={\small\ttfamily},
  numbers=none,
  numberstyle=\tiny\color{gray},
  keywordstyle=\color{blue},
  commentstyle=\color{dkgreen},
  stringstyle=\color{mauve},
  breaklines=true,
  breakatwhitespace=true
  tabsize=3
}




\usepackage[ 
  backref=page,
  pdfpagelabels=true,
  plainpages=false,
  colorlinks=true,
  bookmarks=true,
  pdfview=FitB]{hyperref} % Required for the hyperlinks within the PDF
  
\usepackage{booktabs} % Required for the top and bottom rules in the table
\usepackage{float} % Required for specifying the exact location of a figure or table
\usepackage{graphicx} % Required for including images
\usepackage{lipsum} % Used for inserting dummy 'Lorem ipsum' text into the template

\newcommand{\HRule}{\rule{\linewidth}{0.5mm}} % Command to make the lines in the title page
\setlength\parindent{0pt} % Removes all indentation from paragraphs

%----------------------------------------------------------------------------------------
%	DEFINITION OF EXPERIMENTS
%----------------------------------------------------------------------------------------

\newexperiment{example}{This is an example experiment}
\newexperiment{example2}{This is another example experiment}
\newexperiment{example3}{This is yet another example experiment}
\newexperiment{table}{This shows a sample table}
%\newexperiment{shorthand}{Description of the experiment}

%---------------------------------------------------------------------------------------

\begin{document}

%----------------------------------------------------------------------------------------
%	TITLE PAGE
%----------------------------------------------------------------------------------------

\frontmatter % Use Roman numerals for page numbers
\title{
\begin{center}
\HRule \\[0.4cm]
{\Huge \bfseries Systems Notes \\[0.5cm] \Large Computer Science}\\[0.4cm] % Degree
\HRule \\[1.5cm]
\end{center}
}
\author{\Huge Alex Norton\\ \\ \LARGE ajn123@vt.edu \\[2cm]} % Your name and email address
\date{Beginning 9 February 2014} % Beginning date
\maketitle

\tableofcontents

\mainmatter % Use Arabic numerals for page numbers

%----------------------------------------------------------------------------------------
%	LAB BOOK CONTENTS
%----------------------------------------------------------------------------------------

% Blank template to use for new days:

%\labday{Day, Date Month Year}

%\experiment{}

%Text

%-----------------------------------------

%\experiment{}

%Text

%----------------------------------------------------------------------------------------

\labday{Chapter 8: Shell}

\experiment{Process}
The first thing to understand is that a {\bf computer program} is a passive collection of instructions; a {\bf process} is the actual execution of those instructions.\\

A {\bf process} in user mode is not allowed to execute privileged instructions.  The only way for the process to change from user mode to kernel mode is via an exception such as an {\bf interrupt}, a {\bf fault} or a {\bf trapping system call}.

%-----------------------------------------

\experiment{Context switching} % Multiple experiments can be included in a single day, this allows you to segment what was done each day into separate categories

In computing, a {\bf context switch} is the process of storing and restoring the state (context) of a process so that execution can be resumed from the same point at a later time. This enables multiple processes to share a single CPU and is an essential feature of a multitasking operating system. What constitutes the context is determined by the processor and the operating system.\\

A context switch follows these 3 steps:

\begin{enumerate}
\item   saves the contents of the current process.
\item  restores the saved context of some previously preempted process
\item  passes control to this newly restored process
\end{enumerate}


\experiment{File Descriptor}

A file descriptor is an indicator for a way of accessing a file.


\experiment{Signal}
A {\bf signal} is a message that notifies a process that an event of some type has occured in the system (just like when you press a button on your phone, a message is sent to the operating system).  Each signal corresponds to some kind of system event.  For example, a signal can be used to cancel background jobs.\\



\begin{table}[H]
\begin{tabular}{l l l}
\toprule
\textbf{Integer Value} & \textbf{Name}  \\
\toprule
0 & Standard Input (Stdin)\\
1 & Standard Output (Stdout) \\
2 & Standard Error (Stderr) \\
\bottomrule
\end{tabular}
\caption{Integer Values and their File descriptors}
\label{tab:treatments_xy}
\end{table}


A pending signal is a signal that has been sent but not recieved.  A process can also block a CERTAIN signal.  In this case, blocked sigals can be sent
but will not be received until the signal is unblocked.


\begin{figure}[H] % Example of including images
\begin{center}
\includegraphics[width=0.5\linewidth]{signals}
\end{center}
\caption{Table of possible signals}
\label{fig:example_figure}
\end{figure}

%-----------------------------------------

\experiment{Pipelining}

Pipelining works by setting the standard output(1) of the first command to the standard input(0) of the second command in the pipeline.  
here are a couple of system calls that you may be interested to understand what is happening in more detail, in particular, fork(2), 
execve(2), pipe(2), dup2(2), read(2) and write(2).


\experiment{Fork}

This is an example of the fork method in C.  fork() clones a process from a process.  This new process can be used to execute another process or
do other things.  NOTE:  Once you clone a process you have no idea which order your clones will run in, your code should not depend on the order.  Another thing to remember is that fork returns twice, once in the parent and once in the child (the new process you just created).  The cloned process is exactly the same except for the return value of fork().  In the child process fork will always return 0 and in the parent it will return the process ID of itself so your code can just check for the child process with == 0 and the parent with an else.
\begin{lstlisting}
#include <unistd.h>
#include <stdio.h>

int main(){
int x = 1;
	if (fork() == 0) 
	{// only child executes this
		printf("Child, x = %d\n", ++x);
	} 
	else {
	// only parent executes this
		printf("Parent, x = %d\n", --x);
	}// parent and child execute this
	printf("Exiting with x = %d\n", x);
return 0;}
\end{lstlisting}

%----------------------------------------------------------------------------------------

\labday{Friday, 26 March 2010}

\experiment{table}

\begin{table}[H]
\begin{tabular}{l l l}
\toprule
\textbf{Groups} & \textbf{Treatment X} & \textbf{Treatment Y} \\
\toprule
1 & 0.2 & 0.8\\
2 & 0.17 & 0.7\\
3 & 0.24 & 0.75\\
4 & 0.68 & 0.3\\
\bottomrule
\end{tabular}
\caption{The effects of treatments X and Y on the four groups studied.}
\label{tab:treatments_xy}
\end{table}

Table \ref{tab:treatments_xy} shows that groups 1-3 reacted similarly to the two treatments but group 4 showed a reversed reaction.

%----------------------------------------------------------------------------------------

\labday{Saturday, 27 March 2010}

\experiment{Bulleted list example} % You don't need to make a \newexperiment if you only plan on referencing it once

This is a bulleted list:

\begin{itemize}
\item Item 1
\item Item 2
\item \ldots and so on
\end{itemize}

%-----------------------------------------

\experiment{example}

\lipsum[6]

%-----------------------------------------

\experiment{example2}

\lipsum[7]

%----------------------------------------------------------------------------------------
%	FORMULAE AND MEDIA RECIPES
%----------------------------------------------------------------------------------------

\labday{} % We don't want a date here so we make the labday blank

\begin{center}
\HRule \\[0.4cm]
{\huge \textbf{Formulae and Media Recipes}}\\[0.4cm] % Heading
\HRule \\[1.5cm]
\end{center}

%----------------------------------------------------------------------------------------
%	MEDIA RECIPES
%----------------------------------------------------------------------------------------

\newpage

\huge \textbf{Media} \\ \\

\normalsize \textbf{Media 1}\\
\begin{table}[H]
\begin{tabular}{l l l}
\toprule
\textbf{Compound} & \textbf{1L} & \textbf{0.5L}\\
\toprule
Compound 1 & 10g & 5g\\
Compound 2 & 20g & 10g\\
\bottomrule
\end{tabular}
\caption{Ingredients in Media 1.}
\label{tab:med1}
\end{table}

%-----------------------------------------

%\textbf{Media 2}\\ \\

%Description

%----------------------------------------------------------------------------------------
%	FORMULAE
%----------------------------------------------------------------------------------------

\newpage

\huge \textbf{Formulae} \\ \\

\normalsize \textbf{Formula 1 - Pythagorean theorem}\\ \\
$a^2 + b^2 = c^2$\\ \\

%-----------------------------------------

%\textbf{Formula X - Description}\\ \\

%Formula

%----------------------------------------------------------------------------------------

\end{document}